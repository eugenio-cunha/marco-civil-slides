% Bruno Germano
\begin{frame}{Artigo 24}
Constituem diretrizes para a atuação da União, dos Estados, do Distrito Federal e dos Municípios no desenvolvimento da internet no Brasil:

I - estabelecimento de mecanismos de governança multiparticipativa, transparente, colaborativa e democrática, com a participação do governo, do setor empresarial, da sociedade civil e da comunidade acadêmica;

II - promoção da racionalização da gestão, expansão e uso da internet, com participação do Comitê Gestor da internet no Brasil;

III - promoção da racionalização e da interoperabilidade tecnológica dos serviços de governo eletrônico, entre os diferentes Poderes e âmbitos da Federação, para permitir o intercâmbio de informações e a celeridade de procedimentos;

IV - promoção da interoperabilidade entre sistemas e terminais diversos, inclusive entre os diferentes âmbitos federativos e diversos setores da sociedade;

V - adoção preferencial de tecnologias, padrões e formatos abertos e livres;

VI - publicidade e disseminação de dados e informações públicos, de forma aberta e estruturada;

VII - otimização da infraestrutura das redes e estímulo à implantação de centros de armazenamento, gerenciamento e disseminação de dados no País, promovendo a qualidade técnica, a inovação e a difusão das aplicações de internet, sem prejuízo à abertura, à neutralidade e à natureza participativa;

VIII - desenvolvimento de ações e programas de capacitação para uso da internet;

IX - promoção da cultura e da cidadania; e

X - prestação de serviços públicos de atendimento ao cidadão de forma integrada, eficiente, simplificada e por múltiplos canais de acesso, inclusive remotos.
\end{frame}